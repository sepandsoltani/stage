\chapter{Discussion}
The Bayesian Geometric Transfer Matrix (BGTM) method demonstrated significant improvements in IDIF accuracy compared to classical GTM approaches through enhanced handling of partial volume effects.
This advancement was achieved by integrating TOF-MRA-guided carotid segmentation with population-based AIF prior knowledge within a Bayesian framework.

The refined segmentation algorithm, particularly through implementation of the cuboid mask, proved crucial for processing challenging data from comatose patients who frequently present brain lesions (Figure~\ref{fig:seg_compare}).
However, occasional over-conservatism in vessel selection suggests the region-growing and thresholding parameters can be optimized to accurately capture the complete carotid structure.
The lack of reference or ground truth carotid mask prevented the evaluation of the segmentation.
This also disabled us from accurately evaluating the proposed IDIF algorithm, as the calculated errors are accumulated with the segmentation error as well.
With labeled data, deep learning approaches such as U-Net can also be explored to improve segmentation accuracy \cite{ronneberger2015u}.


The cAUC error reduction (Figure~\ref{subfig:fdg_cauc_boxplot}) confirms Bayesian priors effectively constrain AIF estimates to physiological ranges.
The strong correlation between cAUC errors and quantification errors (Figure~\ref{fig:corr_mat}) validates cAUC as a robust intermediate metric for IDIF assessment, enabling rapid validation without requiring full kinetic modeling.

Quantitative analysis revealed strong agreement between BGTM and AIF-derived $\mrglu$ values, with the Bayesian approach achieving a 57\% reduction in absolute error compared to conventional GTM methods (Table~\ref{tab:metrics}).
Despite this improvement, the observed MAPE results exhibited substantial dispersion (Coefficient of Variation = 71\%).
This inconsistency is particularly evident in outlier cases where BGTM underperformed (Figure~\ref{fig:fdg_ifs}).
Although, further investigation is required to fully characterize the underlying causes, one plausible factor could be TOF-MRA/PET misregistration resulting in inaccurate TAC extraction.


Our analysis was limited to a comatose patient cohort from a single imaging center, which may limit generalizability to populations with normal cerebral blood flow.
Future validation should incorporate multi-center studies involving both healthy subjects and different radiotracers to establish broader applicability.

Additionally, the current PCA implementation of randomly selecting 10 subjects from the population lacks practical viability for standardized implementation.
Future iterations should only be limited to a predefined population.
