\chapter*{Abstract}
\addcontentsline{toc}{chapter}{Abstract}
Dynamic PET quantification requires an arterial input function (AIF), but arterial sampling is invasive.
Image-derived input functions (IDIFs) offer a non-invasive alternative, yet accurate segmentation of arteries and partial volume effects (PVE) limit accuracy.
We present an automatic pipeline that segments the internal carotid arteries on TOF-MRA and estimates an IDIF with a Bayesian Geometric Transfer Matrix (BGTM) model.
The model uses a population-informed PCA prior for the input function, a spectral model for background tissue, and frame-wise weights to account for time-varying noise.
We evaluated the approach on two experimental cohorts: comatose patients with an injection of \fdg\ and healthy adults with an injection of \yohimbine.
In the \fdg\ dataset, BGTM improved quantitative accuracy over GTM and a population-based input function (PBIF), reducing \(\mrglu\) error on average (mean MAPE 13\% vs 24\% for GTM and 17\% for PBIF).
In the \yohimbine\ dataset, BGTM lowered the large errors seen with GTM but did not surpass PBIF (mean \(V_T\) MAPE 75.9\% vs 166.5\% and 29.5\%).
Performance was likely affected by the small PCA sample and by subject motion.
We also conducted Monte Carlo PET simulations as complementary validation under controlled conditions, with known ground truth.
Simulations reproduced realistic brain activity but showed a systematic underestimation of arterial activity relative to the prescribed input.
Indicating the need for a more rigorous simulation design in future studies
These results show that non-invasive IDIF is feasible with multimodal PET--MR and principled treatment of PVE and noise, with clear gains for \fdg\ and potential for higher accuracy for other tracers as well.
\paragraph{Keywords:} Dynamic FDG-PET, Image-Derived Input Function, Hybrid PET/MRI, Bayesian Framework

