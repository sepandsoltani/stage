\chapter*{Abstract}
\addcontentsline{toc}{chapter}{Abstract}

Positron Emission Tomography (PET) combined with kinetic modeling enables quantitative assessment of physiological processes through the use of an arterial input function (AIF).
Traditional AIF acquisition via arterial sampling is invasive and challenging, particularly in comatose patients.
This study aimed to enhance a non-invasive Bayesian Geometric Transfer Matrix (BGTM) method for estimating image-derived input function (IDIF) using Time-of-Flight Magnetic Resonance Angiography (TOF-MRA) acquired simultaneously with dynamic PET on a hybrid PET/MR scanner.
The proposed method integrates TOF-MRA-guided carotid segmentation with a Bayesian framework incorporating population-based AIF priors to address partial volume effects.
Evaluated on a cohort of 56 comatose patients, the refined segmentation pipeline, incorporating a cuboid mask to exclude non-carotid tissues, improved robustness in challenging cases.
The BGTM approach demonstrated a 57\% reduction in mean absolute error in the cumulative Area Under the Curve (cAUC) errors (14,202 vs. 33,764 vs, $p<0.0001$) and a signifact reduction (14.1\% vs. 33.0\%, $p<0.0001$) in mean absolute percentage error (MAPE) for glucose metabolic rate ($\mrglu$) quantification compared to conventional GTM method.
However, high variability in $\mrglu$ and cAUC errors(71\% and 65\% coefficient of variation, respectively) highlights inconsistencies in the method.
These results position BGTM as a promising pathway toward reducing reliance on invasive arterial sampling, though addressing residual variability remains critical for future clinical applications.

\paragraph{Keywords:} Dynamic FDG-PET, Image-Derived Input Function, Hybrid PET/MRI, Bayesian Framework

