\chapter{Conclusion}
We presented an automatic pipeline for non-invasive input-function estimation in brain PET.
The approach combines TOF-MRA-based ICA segmentation with a Bayesian GTM model that places priors on the input function and background kinetics and accounts for time-varying noise.
On the \fdg\ dataset, it significantly improved \(\mrglu\) quantification compared with GTM and PBIF.
For \yohimbine, it reduced the large GTM errors but did not surpass PBIF, likely due to motion during scanning and the small PCA cohort.
As complementary validation, we ran simulations; the results were suboptimal and warrant further investigation into their failure modes.

\section*{Career Prospects}
In this internship, I developed a deeper understanding and a strong interest in research in PET kinetic modelling and medical image processing.
Therefore, I will pursue a PhD at CERMEP under the supervision of Dr. Costes and Dr. Merida after completing this Master's program.
Funding was secured through the annual PhD fellowship competition of the Interdisciplinary Doctoral School of Health Sciences (L'École Doctorale Interdisciplinaire Sciences Santé, EDISS), in which I placed joint first.

The PhD is titled “Multimodal PET–MR Imaging: Dynamic Modelling of Neurotransmitter Release and Analysis of Neurovascular Coupling.”
The project aims to implement and validate a kinetic modelling method that enables detection and characterization of endogenous neurotransmitter release, in competition with a specific PET tracer (ntPET), and to link this release to neurovascular coupling measured with functional MRI.

At CERMEP, a Bayesian framework for estimating kinetic model parameters has been developed \cite{irace2020bayesian} which is actually the inspiration of the original BGTM implementation and this internship.
The performance of our approach has been evaluated on simulated data.
However, further work is required to develop a robust statistical model able to assess the likelihood associated with a discharge and improve the estimation by incorporating prior knowledge from functional MRI data.
