\chapter{Introduction}
\section{Positron Emission Tomography}
Positron Emission Tomography (PET) is an in vivo functional imaging technique widely used in clinical and research settings to monitor physiological and biochemical processes.
In PET, a biologically active molecule is labeled with a positron-emitting radioisotope, serving as a radiotracer, and then injected into the body.
As the radiotracer accumulates in target tissues, its radioactive decay produces positrons, which interact with electrons to emit pairs of gamma photons in nearly opposite directions.
These photons are detected by the PET scanner, and image reconstruction algorithms generate a three-dimensional representation of the tracer distribution.
This imaging modality allows for the investigation of metabolic changes, receptor binding, and other biochemical processes, providing invaluable information in oncology, neurology, cardiology, and other fields.

There are two main categories in PET image acquisition: static imaging and dynamic imaging.
Static PET involves acquiring a single scan after the radiotracer injection.
This single snapshot offers a powerful yet simplified view of tracer distribution.
The common quantification metric in static imaging is the Standardized Uptake Value (SUV), which normalizes tissue uptake by the injected dose and weight of the subject, allowing for a semi-quantitative comparison of tracer accumulation across different tissues or over time \cite{keyes1995suv}.
Due to its simplicity, static PET is widely used in clinical settings; however, it also has limitations.
Because it reflects only one time point, the SUV cannot capture the temporal dynamics of tracer uptake and clearance, and various physiological factors may influence its measurements, thereby reducing its accuracy.

Dynamic PET imaging provides a more comprehensive view of radiotracer kinetics by acquiring a series of images over a period ranging from a few minutes to more than an hour post-injection, depending on the tracer type.
Instead of a single snapshot, dynamic imaging produces time-activity curves (TACs) that illustrate how tracer concentration in each tissue changes throughout the scanning period.
This approach enables the measurement of physiological parameters such as the tracer rate of influx (\(K_i\)) for radiotracers with irreversible uptake (e.g. \fdg ), volume of distribution (\(V_T\)), and the rates of phosphorylation and dephosphorylation.

\section{Kinetic Modeling}
To quantify pharmacokinetic parameters, kinetic modeling is employed.
Compartmental modeling is the most popular and is considered the most accurate approach in kinetic modeling.
In compartmental modeling, the distribution and kinetics of a radiotracer are described by dividing the system into distinct compartments, each representing a pool of tracer that behaves uniformly.
Interactions between compartments can be unidirectional or bidirectional, meaning the tracer may either move in and out or only enter a compartment.
Various graphical models (e.g., the Logan \cite{logan1990graphical} and Patlak \cite{patlak1983graphical} methods), as well as classical compartmental model fitting approaches, are used to analyze tracer kinetics.

Figure~\ref{fig:2tcm} shows the two-tissue compartment model (2TCM), also known as the three-compartment model, in series mode.
This model comprises one tissue compartment for the free tracer, \(C_F(t)\), and another for the receptor-bound tracer, \(C_B(t)\), in addition to an external compartment representing the tracer concentration in the plasma or blood, denoted as the input function \(C_P(t)\).

The tracer kinetics are governed by a series of first-order differential equations, in which the exchange rates between the compartments are considered constant:
\begin{align}
	\frac{dC_F(t)}{dt} & = K_1 \, C_P(t) \;-\; \bigl(k_2 + k_3\bigr) C_F(t) \;+\; k_4 C_B(t) \,, \label{eq:2tcm-c1} \\[6pt]
	\frac{dC_B(t)}{dt} & = k_3 \, C_F(t) \;-\; k_4 \, C_B(t), \label{eq:2tcm-c2}
\end{align}
where \(K_1\), \(k_2\), \(k_3\), and \(k_4\) are the constant rate parameters.

\begin{figure}[b]
	\centering
	\begin{tikzpicture}[>=latex, thick]
		\node[draw, rounded corners, minimum width=2.5cm, minimum height=2cm, align=center] (Cp) at (0,0) {$C_P$};
		\node[draw, rounded corners, minimum width=2.5cm, minimum height=2cm, align=center] (C1) at (5,0) {$C_F$};
		\node[draw, rounded corners, minimum width=2.5cm, minimum height=2cm, align=center] (C2) at (10,0) {$C_B$};

		\draw[->]
		([yshift=8pt]Cp.east) to[out=0, in=180]
		node[above] {\(K_1\)}
		([yshift=8pt]C1.west);

		\draw[->]
		([yshift=-8pt]C1.west) to[out=180, in=0]
		node[below] {\(k_2\)}
		([yshift=-8pt]Cp.east);

		\draw[->]
		([yshift=8pt]C1.east) to[out=0, in=180]
		node[above] {\(k_3\)}
		([yshift=8pt]C2.west);

		\draw[->]
		([yshift=-8pt]C2.west) to[out=180, in=0]
		node[below] {\(k_4\)}
		([yshift=-8pt]C1.east);

		\draw[dashed, rounded corners, thick] ($(C1.north west)+(-0.3,0.3)$) rectangle ($(C2.south east)+(0.3,-0.3)$);

	\end{tikzpicture}
	\caption{Schematic of the two-tissue compartment model (2TCM)}
	\label{fig:2tcm}
\end{figure}

The total radiotracer tissular kinetic measured by PET (the PET data), \(C_T(t)\), is given by
\begin{equation}
	C_T(t) \;=\; C_F(t) \;+\; C_B(t) \;+\; C_P(t).
\end{equation}

Thus to solve this system of equations and to estimate \(K_1\), \(k_2\), and \(k_3\) parameters, we must fit the model using the measured PET TACs ($C_T$) and the input function ($C_P$).

For \fdg$\,$ quantification, the metabolic rate of glucose (\(\textrm{MR}_{\textrm{glu}}\)) is calculated as
\begin{equation}
	\textrm{MR}_{\textrm{glu}} \; (\textrm{\textmu mol/min/100g}) = \frac{[C]}{LC} \cdot \frac{K_1 \times k_3}{k_2 + k_3} \,.
\end{equation}
where \([C]\) denotes the glucose concentration, and \(LC\) is the lumped constant.

For \yohimbine $\,$ quantification, we utilize the Volume of Distribution ($V_T$) which is the ratio of radiotracer concentration in the target tissue ($C_T$) to the plasma ($C_P$):
\[
	V_T = \frac{C_T}{C_P}
\]

Using the Logan plot method this can be directly estimated from these two values or to be fitted to a compartment model. In the latter case, $V_T$ can be calculated as
\[
	V_T = \frac{K_1}{k_2} (1+\frac{k_3}{k_4})
\]


\section{Input Function}
\subsection{Arterial Input Function}
% GPT: Dont change
The arterial input function (AIF) is considered the gold standard for obtaining the input function.
It is determined by inserting an arterial catheter into the patient and continuously drawing blood samples to measure the radiotracer concentration, thereby obtaining the blood activity curve used in the quantification model.
However, this procedure is invasive and can cause discomfort, potentially discouraging patients from undergoing future examinations.
Furthermore, this method is labor-intensive and requires trained personnel to manage both the subject and the measurement devices.

\subsection{Image Derived Input Function}
% GPT: Dont change
The image-derived input function (IDIF) has been proposed as a non-invasive alternative for obtaining the input function.
IDIF techniques typically involve identifying vascular structures or regions with high blood pool activity within the imaging field and extracting the input function directly from the PET images.
In brain PET imaging, the carotid arteries are the largest vessels present in the limited field of view (FOV) and have a diameter of approximately 5 mm, which is comparable to the typical spatial resolution of PET (4-6mm FWHM).
Therefore, directly extracting the carotids from PET images is challenging due to the limited resolution and the strong partial volume effects (PVE) present.

\subsection{Population-Based Input Function}
Population-Based Input Function (PBIF) is a method for replacing the subject specifc AIF with the average AIFs of a population of other subjects.
% GPT: explain a bit more, max 1-2 sentences

\subsection{Background}
Many methods have utilized the first few frames of the dynamic PET where tracer only exists in the arteries and has not reached other tissues yet, to perform a high intensity tresholding to obtain the mask of the carotids \cite{young2023image}.
Due to PVE, these methods are not totally accurate therefore many different take clever approaches such as supervised clustering \cite{TODO}, Deep learning approaches \cite{TODO} and [TODO] \ref{TODO}



With the emergence of hybrid PET/MRI machines, it has become feasible to acquire both functional and anatomical data simultaneously.
MRI provides high-resolution soft tissue contrast, while PET captures metabolic activity.
For instance, time-of-flight MR angiography (TOF-MRA) delivers excellent arterial contrast, which allows for accurate segmentation of vascular structures such as the carotid arteries.
However, even with a high-resolution anatomical guidance, directly applying the segmented arterial mask to the PET images introduces challenges \cite{zanotti2011image}.
In particular, the limited spatial resolution of PET can lead to partial volume effects, resulting in inaccuracies in the derived input function.
Consequently, additional correction techniques are required to mitigate these effects and ensure reliable quantification.

Several methods have been developed to enhance IDIF accuracy by addressing partial volume correction and artery segmentation.
PET-only methods aim to extract the artery from the PET image itself which is


Several methods have been developed to enhance IDIF accuracy by addressing partial volume correction and automation challenges.
For partial volume correction,
Early PET-only approaches primarily focused on classical partial volume correction (PVC) techniques \cite{mourik2009image}, while more recent methods have explored supervised clustering algorithms \cite{lyoo2014image} and deep learning approaches \cite{chavan2024end,ferrante2024physically}.
Hybrid PET/MRI methods improve carotid artery delineation by leveraging anatomical imaging. The caliPER software, for instance, employs a semi-automated approach for IDIF extraction using carotid masks obtained from T1 or TOF-MRA images, incorporating partial volume correction \cite{dassanayake2022caliper}.
Additionally, fully automated methods have been proposed to enhance carotid segmentation and PVC techniques \cite{sari2017estimation, jochimsen2016fully, khalighi2018image, sundar2019towards}.

\citeauthor{irace2021bayesian} \cite{irace2021bayesian} proposed an automatique method for IDIF estimation by performing TOF-MRA driven carotid segmentation and using a Bayesian framework for incorporating prior knowledge into a geometric transfer matrix method \cite{rousset1998correction}---a classical partial volume correction technique.
In this work, the aim was to improve this method and enhance accuracy by evaluating it on a dataset of comatose patients.
