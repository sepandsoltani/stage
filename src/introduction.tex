\chapter{Introduction}
\section{Positron Emission Tomography}
% GPT: dont touch
Positron Emission Tomography (PET) is an in vivo functional imaging technique widely used in clinical and research settings to monitor physiological and biochemical processes.
In PET, a biologically active molecule is labeled with a positron-emitting radioisotope, serving as a radiotracer, and then injected into the body.
As the radiotracer accumulates in target tissues, its radioactive decay produces positrons, which interact with electrons to emit pairs of gamma photons in nearly opposite directions.
These photons are detected by the PET scanner, and image reconstruction algorithms generate a three-dimensional representation of the tracer distribution.
This imaging modality allows for the investigation of metabolic changes, receptor binding, and other biochemical processes, providing invaluable information in oncology, neurology, cardiology, and other fields.

There are two main categories in PET image acquisition: static imaging and dynamic imaging.
Static PET involves acquiring a single scan after the radiotracer injection.
This single snapshot offers a powerful yet simplified view of tracer distribution.
The common quantification metric in static imaging is the Standardized Uptake Value (SUV), which normalizes tissue uptake by the injected dose and weight of the subject, allowing for a semi-quantitative comparison of tracer accumulation across different tissues or over time \cite{keyes1995suv}.
Due to its simplicity, static PET is widely used in clinical settings; however, it also has limitations.
Because it reflects only one time point, the SUV cannot capture the temporal dynamics of tracer uptake and clearance, and various physiological factors may influence its measurements, thereby reducing its accuracy.

Dynamic PET imaging provides a more comprehensive view of radiotracer kinetics by acquiring a series of images over a period ranging from a few minutes to more than an hour post-injection, depending on the tracer type.
Instead of a single snapshot, dynamic imaging produces time-activity curves (TACs) that illustrate how tracer concentration in each tissue changes throughout the scanning period.
This approach enables the measurement of physiological parameters such as the tracer rate of influx (\(K_i\)) for radiotracers with irreversible uptake (e.g. \fdg ), volume of distribution (\(V_T\)), and the rates of phosphorylation and dephosphorylation.

\section{Kinetic Modeling}
% GPT: dont touch
To quantify pharmacokinetic parameters, kinetic modeling is employed.
Compartmental modeling is the most popular and is considered the most accurate approach in kinetic modeling.
In compartmental modeling, the distribution and kinetics of a radiotracer are described by dividing the system into distinct compartments, each representing a pool of tracer that behaves uniformly.
Interactions between compartments can be unidirectional or bidirectional, meaning the tracer may either move in and out or only enter a compartment.
Various graphical models (e.g., the Logan \cite{logan1990graphical} and Patlak \cite{patlak1983graphical} methods), as well as classical compartmental model fitting approaches, are used to analyze tracer kinetics.

Figure~\ref{fig:2tcm} shows the two-tissue compartment model (2TCM), also known as the three-compartment model, in series mode.
This model comprises one tissue compartment for the free tracer, \(C_F(t)\), and another for the receptor-bound tracer, \(C_B(t)\), in addition to an external compartment representing the tracer concentration in the plasma or blood, denoted as the input function \(C_P(t)\).

The tracer kinetics are governed by a series of first-order differential equations, in which the exchange rates between the compartments are considered constant:
\begin{align}
	\frac{dC_F(t)}{dt} & = K_1 \, C_P(t) \;-\; \bigl(k_2 + k_3\bigr) C_F(t) \;+\; k_4 C_B(t) \,, \label{eq:2tcm-c1} \\[6pt]
	\frac{dC_B(t)}{dt} & = k_3 \, C_F(t) \;-\; k_4 \, C_B(t), \label{eq:2tcm-c2}
\end{align}
where \(K_1\), \(k_2\), \(k_3\), and \(k_4\) are the constant rate parameters.

\begin{figure}[b]
	\centering
	\begin{tikzpicture}[>=latex, thick]
		\node[draw, rounded corners, minimum width=2.5cm, minimum height=2cm, align=center] (Cp) at (0,0) {$C_P$};
		\node[draw, rounded corners, minimum width=2.5cm, minimum height=2cm, align=center] (C1) at (5,0) {$C_F$};
		\node[draw, rounded corners, minimum width=2.5cm, minimum height=2cm, align=center] (C2) at (10,0) {$C_B$};

		\draw[->]
		([yshift=8pt]Cp.east) to[out=0, in=180]
		node[above] {\(K_1\)}
		([yshift=8pt]C1.west);

		\draw[->]
		([yshift=-8pt]C1.west) to[out=180, in=0]
		node[below] {\(k_2\)}
		([yshift=-8pt]Cp.east);

		\draw[->]
		([yshift=8pt]C1.east) to[out=0, in=180]
		node[above] {\(k_3\)}
		([yshift=8pt]C2.west);

		\draw[->]
		([yshift=-8pt]C2.west) to[out=180, in=0]
		node[below] {\(k_4\)}
		([yshift=-8pt]C1.east);

		\draw[dashed, rounded corners, thick] ($(C1.north west)+(-0.3,0.3)$) rectangle ($(C2.south east)+(0.3,-0.3)$);

	\end{tikzpicture}
	\caption{Schematic of the two-tissue compartment model (2TCM)}
	\label{fig:2tcm}
\end{figure}

The total radiotracer tissular kinetic measured by PET (the PET data), \(C_T(t)\), is given by
\begin{equation}
	C_T(t) \;=\; C_F(t) \;+\; C_B(t) \;+\; C_P(t).
\end{equation}

Thus to solve this system of equations and to estimate \(K_1\), \(k_2\), and \(k_3\) parameters, we must fit the model using the measured PET TACs ($C_T$) and the input function ($C_P$).

For \fdg$\,$ quantification, the metabolic rate of glucose (\(\textrm{MR}_{\textrm{glu}}\)) is calculated as
\begin{equation}
	\textrm{MR}_{\textrm{glu}} \; (\textrm{\textmu mol/min/100g}) = \frac{[C]}{LC} \cdot \frac{K_1 \times k_3}{k_2 + k_3} \,.
\end{equation}
where \([C]\) denotes the glucose concentration, and \(LC\) is the lumped constant.

For \yohimbine $\,$ quantification, we utilize the Volume of Distribution ($V_T$) which is the ratio of radiotracer concentration in the target tissue ($C_T$) to the plasma ($C_P$):
\[
	V_T = \frac{C_T}{C_P}
\]

Using the Logan plot method this can be directly estimated from these two values or to be fitted to a compartment model. In the latter case, $V_T$ can be calculated as
\[
	V_T = \frac{K_1}{k_2} (1+\frac{k_3}{k_4})
\]


\section{Input Function}
\subsection{Arterial Input Function}
% GPT: Dont change
The arterial input function (AIF) is considered the gold standard for obtaining the input function.
It is determined by inserting an arterial catheter into the patient and continuously drawing blood samples to measure the radiotracer concentration, thereby obtaining the blood activity curve used in the quantification model.
However, this procedure is invasive and can cause discomfort, potentially discouraging patients from undergoing future examinations.
Furthermore, this method is labor-intensive and requires trained personnel to manage both the subject and the measurement devices.

\subsection{Population-Based Input Function}
Population-Based Input Function (PBIF) is a method for replacing the subject specific AIF with the average AIFs of a population of other subjects.
% GPT: explain a bit more, max 1-2 sentences

\subsection{Image Derived Input Function}
% GPT: Dont change
The image-derived input function (IDIF) has been proposed as a non-invasive alternative for obtaining the input function.
IDIF techniques typically involve identifying vascular structures or regions with high blood pool activity within the imaging field and extracting the input function directly from the PET images.
For example, in whole-body, cardiac or small animal PET studies the aorta is visible in the in the Field of View (FOV) of the PET camera and is used as the source for IDIF.
In brain PET imaging, the Internal Carotid Arteries (ICA) are the largest vessels present in the FOV and have a diameter of approximately 5 mm much smaller than of the aorta.
Their smaller size makes IDIF more challenging due to Partial Volume Effects

\section{Partial Volume Effect}
% GPT: explain PVE in a few lines. 
Partial Volume Effect (PVE) is



\section{Background}
% GPT: improve sentencing and add slightly more description to some methods that i didnt explain enough
Many methods have utilized the first few frames of the dynamic PET which the tracer has not reached tissues yet and mainly exists in the arteries, to obtain the mask of the carotids \cite{young2023image}.
These implementation vary from completely manual \cite{TODO}, semi-automatic by just providing a few seed voxels for a region growing algorithm \cite{TODO} or for the center of a dilation, and completely automatic by using sophisticated deep learning methods such as custom Convolutional Neural Networks (CNN) \cite{TODO} and U-NET \cite{TODO}.

However even with the sophisticated approaches, ICA segmentation directly from PET still suffer from the strong PVE present.
With the emergence of hybrid PET/MRI machines, it has become feasible to acquire both functional and anatomical data simultaneously.
MRI provides high-resolution soft tissue contrast, while PET captures metabolic activity.
For instance, time-of-flight MR angiography (TOF-MRA) delivers excellent arterial contrast.
Many methods have taken advantage of this fact to deliver simple yet very accurate segmentation of the ICA by manual or semi-automatic procedures usually involving high intensity thresholding or seeded or not-seeded region growing algorithms \cite{TODO}.

Unlike T1-weighted MRI, TOF-MRA is not usually available in studies thus \citeauthor{TODO} \cite{TODO} proposed a deep learning model to extract the ICA from the T1-weighted images.
\cite{TODO} in the caliPER software enabled the semi-automatic extraction of the arteries from both TOF-MRA as well as T1-weighted images.

However, as previously mentioned, even with a high-resolution anatomical mask of the arteries, the issue of PVE still exists and must be accounted for.
Many Partial Volume Correction (PVC) methods involve in deriving the spill-in and spill-out coefficient of the ICA and the surrounding tissue.
Recovery Coefficient (RC) is the most popular method of estimating the spill-in and spill-out coefficient \cite{TODO,TODO,TODO,TODO}.
They are computed by scanning cylindrical phantoms of different diameters and matching the diameter of the ICA to them \cite{TODO}.
Geometric Transfer Matrix (GTM) is a more general method adapted to any type of volume by estimating spillage coefficients by based on the geometry of the VOI and modelling the PSF of the PET\cite{TODO}.
However, these methods do not account for the apparent noise in the PET and% GPT: add one more reason. like i wanna say its just a simple linear mixing like SP*C1-SO*C2   
thus they fail to fully recover the lost signal due to noise.

More complex methods have been proposed such as model-based matrix factorization to estimate the input function and surrounding tissue activity \cite{TODO} and deep learning approaches \cite{TODO,TODO}

In this work, we propose a fully automatic pipeline for extracting the ICA mask from the TOF-MRA.
We utilize a Bayesian framework to model the input function and the surrounding tissue based on their interaction according to GTM by incorporating prior knowledge and also accounting for noise \cite{irace2021bayesian}.
